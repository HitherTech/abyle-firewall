%------------------------------------------------------------------------------
%	Abyle Installation
%------------------------------------------------------------------------------
\chapter{Installation}

\section{Prerequisites}

\subsection{Debian}

python ($>=$ 2.3)\newline
python-xml\newline
iptables\newline

\subsection{Ubuntu}

python ($>=$ 2.3)\newline
python-xml\newline
iptables\newline

\subsection{Fedora}

\subsection{Suse}

\subsection{Gentoo}

\section{Installation with abyle-installer.sh}
untar your abyle package usually named abyle-$<$VERSION$>$.tar.bz2 with the following command\dnl
\courierfont
tar xvfj abyle-$<$VERSION$>$.tar.bz2\dnl
\normalfont
if your tar doen't support bzip2 decompression:\dnl
\courierfont
bzcat abyle-$<$VERSION$>$.tar.bz2 | tar xvf -\dnl
\normalfont
Now you can change the directory to ./abyle-pkg\dnl
\courierfont
cd abyle-pkg\dnl
\normalfont
Now run the Abyle-Installer with:\dnl
\courierfont
./abyle-installer.sh install\dnl
\normalfont
\begin{itemize}
\item The installer will ask you for the path to your python site-package directory, this is the place where your python modules (classes) are located - on debian based distributions it's usally:\dnl
\courierfont
/usr/local/lib/python$<$VERSION$>$/site-packages\dnl
\normalfont
\item The next question is about where you want to locate the abyle main script usually:\dnl 
\courierfont
/usr/local/sbin\dnl
\normalfont
or\dnl
\courierfont
/usr/sbin\dnl
\normalfont
\item The last question is about the configuration directory if you choose to place it on a non default place you have to mention to put the \courierfont--config-path=$<$\$PATH$>$\normalfont parameter to every abyle execution.
If you have completed all steps successfully you can get further informations by typing:\dnl
\courierfont
\# abyle --help\dnl
\normalfont
The abyle-main script looks for a suitable python intepreter in your \$PATH by invoking: /usr/bin/env, so if you want to use another installed python version u have to replace the first line of the script with the path to your desired python intepreter or you have to make a softlink in the path to this version. On Debian Systems it's most common to have at least two different python version installed.\newline
Bevore you can use abyle u have to install a default template configuration for each interface,
this could be done by executing:\dnl
\courierfont
\# abyle -t\dnl
\normalfont
\end{itemize}

\section{Manual Installation}



